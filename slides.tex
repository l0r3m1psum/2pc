\documentclass[landscape]{slides}
\usepackage{lscape}

\newcommand{\slidetitle}{}
\newcommand{\settitle}[1]{\renewcommand{\slidetitle}{#1}}

\makeatletter
\LARGE
% To avoid warnings from the font of \slidetitle beeing too big
\setlength{\headheight}{\f@size pt}
\def\ps@mystyle{%
	\def\ps@slide{%
		\def\@oddhead{\hrulefill{\LARGE\centering\slidetitle}\hrulefill}%
		\def\@oddfoot{\@mainsize \mbox{}\hfil\hb@xt@3em{\theslide\hss}}%
		\def\@evenhead{\@oddhead}%
		\def\@evenfoot{\@oddfoot}%
	}%
	\def\ps@overlay{%
		\def\@oddhead{}%
		\def\@oddfoot{\@mainsize \mbox{}\hfil\hb@xt@3em{\theoverlay\hss}}%
		\def\@evenhead{\@oddhead}%
		\def\@evenfoot{\@oddfoot}%
	}%
	\def\ps@note{%
		\def\@oddhead{}%
		\def\@oddfoot{\@mainsize \hfil\thenote}%
		\def\@evenhead{\@oddhead}%
		\def\@evenfoot{\@oddfoot}%
	}%
}
\makeatother
\pagestyle{mystyle}

\title{Model Checking Two Phase Commit}
\author{Diego Bellani}
\date{2022}
\setlength{\unitlength}{1cm}

\begin{document}

\maketitle

\begin{slide}
\settitle{The Problem}
In distributed transaction systems it is necassary to ensure \emph{consistency}
of the executed transactions.

That is a single single logical action (Commit or Abort) is carried out by all
the involved processes.
\end{slide}

\begin{slide}
\settitle{The Two Phase Commit Protocol}
\begin{itemize}
\item It is the simplest most popular atomic commmit protocol.
\item Described in the book ``Concurrency Control and Recovery in Database
	Systems'' by Philip A. Berstein, Vassos Hadzilacos and Nathan Goodman.
\end{itemize}
\end{slide}

\begin{slide}
\settitle{What's an Atomic Commit Protocol?}
Roughly speaking, an ACP is an algorith in which either all processes commit or
they all abort.

More preciselly it is a protocol that respects the following properties\dots

\begin{overlay}
\begin{description}
\item[AC1] All processes that reach a decision reach the same one.
\item[AC2] A process cannot reverse its decision after it has reached one.
\item[AC3] The Commit decision can only be reached if \emph{all} processes voted
	Yes.
\item[AC4] If there are no failures and all processes voted Yes, then the
	decision will be to Commit.
\item[AC5] If all existing failures are repaired and no new failures occur for
	sufficiently long, then all processes will eventually reach a decision.
\end{description}
\end{overlay}
\end{slide}

\begin{slide}
\settitle{A Brief Description}
There are two kind of processes, a \emph{coordinator} and the
\emph{partecipants}.

\begin{center}
\begin{picture}(20,10)
\put(4,9){Coordinator}       \put(12,9){Partecipants}
\put(6,8.5){\vector(0,-1){7}}\put(14.5,8.5){\vector(0,-1){7}}
\put(8.5,8){\texttt{vote\_req}}
\put(6.5,7.5){\vector(1,0){7.5}}
\put(8.5,6){\texttt{yes/no}}
\put(14,5.5){\vector(-1,0){7.5}}
\put(8.5,4){\texttt{commit/abort}}
\put(6.5,3.5){\vector(1,0){7.5}}
\end{picture}
\end{center}
\end{slide}

\begin{slide}
\begin{center}
\LARGE Model Checking
\end{center}
\end{slide}

\begin{slide}
\settitle{The Promela Model}
\begin{quote}
\begin{verbatim}
active [N] proctype partecipant() {
    ...
}
active proctype coordinator() {
    ...
}
ltl AC1 {
    ...
}
...
ltl AC5 {
    ...
}
\end{verbatim}
\end{quote}
\end{slide}

\begin{slide}
\settitle{NuSMV Model}
\begin{quote}
\begin{verbatim}
MODULE main
  VAR
    channel : {empty, vote_req, yes, no, commit, abort};
    coordinator : process coordinator_process(channel);
    partecipant : process partecipant_process(channel);
    ...
\end{verbatim}
\end{quote}
\end{slide}

\begin{slide}
\settitle{Experience Report}
In Promela you have to program much more to expose the property that you care to
check. But the way in which you can describe your model is much more powerfull.

In NuSMV the way in which you can express you model is less
powerfull\footnote{The \texttt{process} keyword is even gettting deprecated!}.
But you have a better way to express your properties.
\end{slide}

\begin{slide}
\settitle{What could have been done better?}
The Cooperative Termination maybe could have been implemented with the help of
some fairness constrains.
\end{slide}

\begin{slide}
\settitle{In Conclusion}
\centering Model checking is really cool.
\end{slide}

\end{document}
